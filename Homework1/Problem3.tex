\documentclass[a4paper,12pt]{article}
\usepackage{amsmath}

\begin{document}
\title{Computational Physics HW1 pro3}
\author{Jiawei Wang}
\date{}
\maketitle

\noindent Since $f_0=2, f_n=f_{n-1}^2$, we have $f_1=4, f_2=16, \cdots, f_n=2^{2^n}$. And now each byte can store 4 bits, we can calculate the maximum value for different data types.\\
a)For int, having 8 bits, the maximum value it can store is $2^7-1<2^{2^3}$, so the maximum value for $f_n$ is $f_2=2^{2^2}$ which means the maximum $n$ is $n=2$.\\
b)For long int, having 16 bits, the maximum value it can store is $2^15-1<2^{2^4}$, so the maximum value for $f_n$ is $f_3=2^{2^3}$ which means the maximum $n$ is $n=3$.\\
c)For unsigned long int, the maximum value it can store is $2^16-1<2^{2^4}$, so the maximum value for $f_n$ is the same as the condition in long int and the maximum $n$ is $n=3$.
\end{document}
